\documentclass[a4paper,11pt]{report}
\usepackage[T1]{fontenc} % pour écrire en français
\usepackage[francais]{babel} %pour écrire en français
\usepackage[utf8x]{inputenc} %encodage en UTF-8
\usepackage{fancyhdr} %pour gérer les en-têtes et pieds de page
\usepackage{amsmath,amscd,amssymb} %pour insérer des expressions scientifiques
\usepackage[pdftex]{graphicx} %pour inclure des figures
\usepackage{subfig}
\usepackage{hyperref} %pour créer des liens hyper-textes
\usepackage{verbatim} %pour citer du code Latex ou autre
\usepackage{url} %pour citer une adresse web
\pagenumbering{arabic} %type de numérotation des pages
\graphicspath{{Figures/}} %les figures sont rangées dans le dossier Figures
\pagestyle{plain} %style des pages

%%%%%%%%%%%%%%%%%%%%%%%%%%%%%%%%%%%%%%%%%%%%%%%%%%%%%%%%%%%%%%%%%%%%

%----------------------------------------------------------------------------------------------------------
%				PAGE DE GARDE
%----------------------------------------------------------------------------------------------------------

\title{Développement durable}
\date{Année universitaire :  2014-2015}
\author{Thomas Gastineau}
 


%%%%%%%%%%%%%%%%%%%%%%%%%%%%%%%%%%%%%%%%%%%%%%%%%%%%%%%%%%%%%%%%%%%%%%%%%%%%%%%%%%%%%%%%%%%%%%%%%%%%%%%%%%%%%%%%
\begin{document}
\maketitle  %génère la page de garde
\newpage  %comme son nom l'indique ...

%%%%%%%%%%%%%%%%%%%%%%%%%%%%%%%%%%%%%%%%%%%%%%%%%%%%%%%%%
\pagenumbering{roman} \setcounter{page}{1} %les pages commencent à être numérotées en lettre romaines.

%%%%%%%%%%%%%%%%%%%%%%%%%%%%%%%%%%%%%%%%%%%%%%%%%%%%%%%%%

\newpage
\null
\thispagestyle{empty}
\newpage

 %------------------------------------------------------------------------------------------------------
 %					    TABLE DES MATIÈRES 
 %-----------------------------------------------------------------------------------------------------
{\tableofcontents} 
%\newpage\
\listoffigures


\newpage

\chapter*{L'Economie Circulaire}
\addcontentsline{toc}{chapter}{L'Economie Circulaire}
\section*{Introduction}
\addcontentsline{toc}{section}{Introducion}
 %Ce chapitre n'est pas numéroté mais apparaitra dans la table des matières gràce à cette commande.
\pagenumbering{arabic} \setcounter{page}{1} %Le numéro de page est remis à zéro et la numérotation est en chiffre arabe

Le principe de l'économie circulaire est la réutilisation de la matière et des ressource afin de diminuer la consommation des matières premières et d'éviter de puiser des ressources naturelles, en supprimant le plus possible l'extraction des matières premières e l'élimination des déchets.
%%%%%%%%%%%%%%%%%%%%%%%%%%%%%%%%%%%%%%%%%%%%%%%%%%%%%%%%%%%%%%%%%%%%%%%%%%%%%%%%%%%%%%%%%%%%%%%%%%%%%%%%%%%%%%%%
\begin{figure}[!h]
\begin{center}
\includegraphics[width=1.2\textwidth]{economie_circulaire} 
\caption{\label{donkey}Economie circulaire}
\end{center}
\end{figure}

 


%%%%%%%%%%%%%%%%%%%%%%%%%%%%%%%%%%%%%%%%%%%%%%%%%%%%%%%%%%%%%%%%%%%%%%%%%%%%%%%%%%%%%%%%%%%%%%%%%%%%%%%%%%%%%%%%

\newpage
\section*{Les Terres Rares}
\addcontentsline{toc}{section}{Les Terres Rares}

Les Terres Rares sont utilisées pour divers produits informatiques tels que les ordinateurs, les smartphones,appareils photo, tablettes etc. Ils sont également utilisés pour les moteurs (néodyme) et pour les panneaux  (indium).\\
Ces matériaux ont été découvert au XVIè siècle mais ce n'est seulement qu'à partir du XIXè siècles que leur utilisation commence réellement car l'exploitation de ces terres rares est problématique, la pollution due a cette exploitation est importante.\\
On appelle ces matériaux terres rares car d'une part, la production n'est pas très importantes moins de 200 000 tonnes/an et donc leur valeur économique est élevée; mais d'autre part, il y a de nombreuses crises liées à la disponibilité qui ont de multiples origines.\\
Tout d'abord, il y a les crises conjoncturelle ,qui sont les crises les plus légères, qui sont dues au climat et aux différentes grèves.\\
Ensuite, 


\section*{Les DEEE}
\addcontentsline{toc}{section}{Les DEEE}

Les DEEE, autrement appelés Déchets d'Equipement Electriques et Electroniques, se divisent en plusieurs catégories qui sont:
\begin{itemize}
\item l'électroménager froid
\item l'électroménager hors froid
\item le grand publique, l'informatique et la télécommunication
\item l'outillage et les jouets
\item les lampes\\
\end{itemize}

Ces déchets sont recyclés pour de multiple raison telles que la raison écologique, afin de réduire l'impacte des déchets sur l'environnement, et la raison économique, afin de diminuer le coût de fabrication des objets en utilisant des matières premières secondaires.

\subsection*{Les Acteurs}
\addcontentsline{toc}{subsection}{Les Acteurs}
Il y a 5 acteurs qui sont: 
\begin{itemize}
\item le producteur
\item le distributeur
\item les collectivités locales
\item les ESS
\item l'utilisateur\\
\end{itemize}
\newpage
Le premier acteur des DEEE est le producteur, c'est lui qui fabrique ou introduit l'objet sur le marché national.
Il doit respecter plusieurs directives, créées en 2002 et appliquées à partir de 2006, qui sont la directive Rotls et la directive DEEE. Cette dernière est une loi qui dit que le producteur de l'objet est responsable de son élimination.\\
Les producteur représente, en France, 0.5\% de la récupération des DEEE.
\\

Le deuxième acteur est le distributeur, c'est lui qui vend l'objet en face à face ou à distance via internet. Il récupère également l'éco-contribution qu'il reverse au producteur. Il a l'obligation de reprendre les déchets des produits endommagés ou anciens que les utilisateur ramènent.\\
En France, il y a à peu près 19500 points de collectes chez les distributeurs. Ils représentent environ 30\% de la récupération des DEEE.
\\

La troisième catégorie d'acteur rassemble les collectivités locales, c'est à dire les déchetteries. Ces dernière sont entretenues avec l'argent que le consommateur a payé dans ses impôts.
Cette catégorie représente 64.5\% de la récupération des DEEE soit 3900 point de collectes.
\\

Le quatrième acteur est l'ESS(Economie Sociale Solidaire). Cette dernière regroupe les association comme EMMAÜS et ENVIE; et représente 5\% de la récupération des DEEE en France soit à peu près 204 points de collecte.
\\

Le dernier acteur est bien entendu l'utilisateur. Il achète le produit et paye en même temps l'éco-contribution. Puis lorsqu'il n'en veut plus ou que ce dernier n'est plus en état, il a le devoir  le DEEE dans un point de collecte. 


\subsection*{Les Eco-organismes}
\addcontentsline{toc}{subsection}{Les Eco-organismes}

\end{document}